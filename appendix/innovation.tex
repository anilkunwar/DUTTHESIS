\chapter*{Abstract of Innovation Points}
\addcontentsline{toc}{chapter}{Abstract of Innovation Points}
\addcontentsline{toe}{chapter}{Abstract of Innovation Points}
The abstract of innovation points in this dissertation include the following:
\begin{enumerate}
    %\item {Propose an ASNET middleware framework that integrates layers from sensing level to application level. It helps to provide the background for design and evaluation of the middleware protocols.}
    \item {Propose ComPAS - a replica allocation mechanism to maximize data availability in ASNETs. We presented the significance of exploiting social features such as friendship and group mobility modeling in allocating replicas of user data. The main aim of this replication protocol is to find an effective and reliable way to store fixed numbers of replicas for each user's data on the storage space of other communities. The decision on the number of replica for each user depends on the replication budget and its desired availability or accessibility. By placing the replica at the storage space of most neighbors, they benefit from this replica when they issue a read query.}
    \item{Propose a community-based load balancing and fairness approach called Co-Lab, which enhances the performance of the data management middleware. This solution exploits offloading and filter replication at the inter-community and intra-community levels, respectively. This facilitates the dynamic dissemination and forwarding of pub/sub services among brokers. Moreover, the exploitation of interest similarity and integration filter-based functionality within each multicast group helps to achieve fair load distribution among each broker and reduce the overall load distribution.}
    \item{Propose BoDMaS - a biologically inspired algorithm to detect and mitigate the impact of selfish users in replication process. Its design integrates user willingness with bacteria social coordination mechanism that aims to identify and counterpart selfishness in ASNETs. BoDMaS performs continuous user assessment, classification, selection of users to detect, and take actions to deny the involvement of selfish users in communication. As a proof of concept and for performance evaluation, BoDMaS is employed in a simulated academic conference event to offer autonomy to mobile participants in identifying and excluding selfish users from their network.}
\end{enumerate}
