\chapter{Conclusion and Future Work}\label{Chap7}
\echapter{Conclusion and Future Work}

\section{Conclusion}\label{Chap7_01}
\esection{Conclusion}
Data management in ASNETs is challenging because it is different from the traditional data management in wired networks in that disconnections are the norm instead of the exception. In this dissertation, we have covered data management protocols in ASNETs with the exploitation and adaptation of different social features and techniques. First, we proposed an ASNET middleware framework which can be considered as an insight we revealed to serve as a basis for the design of middleware protocols in this emerging paradigm. There are no standard design approaches used to design middleware protocols in ASNETs. This dissertation defines and describes a number of approached and techniques that can be used for the design of efficient protocols. Then, we provided three data management protocols for ASNETs, which maximizes data availability, balances the load fairly and excludes selfish users, respectively.

Specifically, we proposed a data replication protocol based on partitioning of social community combined with social relationship and a user level replication so that data availability for all users is guaranteed. The system gives a fixed number of replicas required for each user that results in an efficient replication solution, particularly in terms of read cost. We also introduced the importance of integrating filter-based and multicast-based publish/subscribe systems for designing a community-based load balancing method. We have specifically proposed a community-based load balancing method combined with fault-tolerance techniques that minimizes overloaded brokers by distributing the event publication load among brokers. We employ an offloading mechanism and interest based similarity filter replication, to reduce the overall load distribution among communities and distribute and balance the load among brokers in a community, respectively. Finally, we proposed a bio-inspired protocol to detect and classify users to be either selfish or cooperative in ASNETs. This protocol gives each user the autonomy to take action against selfish users and it exploits out data replication protocol as a data management model. Simulation evaluations are conducted using synthetic traces to show that the proposed protocols are comparatively better than the existing one.

\section{Future Work}\label{Chap7_02}
\esection{Future Work}
Looking into the future, I will continue the research on the ASNET middleware problems. We believe that, there is still a lot that needs to be investigated, social awareness, learning and acquisition levels of the middleware framework, for example. To this end, we go one step further to envisage more state-of-the-art middleware solutions to be developed in the future. Extensions of this work should focus on two major research points.

\begin{enumerate}
    \item {Design a filter replication mechanism to regulate communication overhead while achieving reliability using fault-tolerance. We will also work on reducing misbehaving nodes in a community as they pose a profound negative impact on load balancing. This would allow a more uniform, efficient and reliable service. Co-Lab can also be adapted to integrate considerations for the mobility of nodes in the community, so as to further reduce the overhead incurred by the links between brokers and subscribers.}
    \item {Investigate the data management scheme for different parameters such as maximum speed and percentage of selfish users to enhance the protocol. A possible future work can be on the consideration of malicious users that inject malicious data to the process and evaluate the effectiveness of our system under expanded network environments. It is also possible to consider the load of users in the model so that they will have fair distribution of operation requests.}
\end{enumerate}

Another interesting future research direction would be data forwarding and routing protocols which would consider users online and dynamic social behaviors. Most of the existing socially-aware routing protocols prioritize interest similarity to enhance the performance of the protocol. Such protocols could be less efficient than routing protocols that consider users dynamic behavior in networks with high user mobility or high traffic rates. One example of future work can be on the coordination of the data replication, load balancing and routing mechanisms and investigates resource allocation and overhead among them, which has not been studied before.

