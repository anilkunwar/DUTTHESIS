%%%%%%%%%%%%%%%%%%%%%%%%%%%%%%%%%%%%%%%%%%%%%%%%%%%%%%%%%%%%%%%%%%%%%%%%%%%%%%%%
%
%   Doctoral Dissertation of Dalian University of Technology
%   By Jie Li and Ahmedin Mohammed Ahmed on the work of Hui Wang
%   package.tex
%   Version: 0.1
%   Update: 2014-04-16
%   Environment: Windows 8.1 + CTeX 2.9.2.164 + WinEdt 7.0
%   欢迎使用,如有任何意见和建议请发送邮件至 yuri_1985@163.com,感谢您的支持!
%
%%%%%%%%%%%%%%%%%%%%%%%%%%%%%%%%%%%%%%%%%%%%%%%%%%%%%%%%%%%%%%%%%%%%%%%%%%%%%%%%

% 版面控制
\usepackage[body={16.1true cm,22.0true cm}]{geometry}   %论文版芯161mm×220mm(不包括页眉页脚)
\usepackage{indentfirst}    % 首行缩进宏包
\usepackage[sf]{titlesec}   % 控制标题的宏包
\usepackage{titletoc}       % 控制目录的宏包
\renewcommand\bibname{References}
\usepackage{fancyhdr}       % 自定义页眉页脚
\usepackage{fancyref}       % 引用链接属性
\usepackage{cite}           % 支持引用的宏包
\usepackage[perpage,symbol]{footmisc}   % 脚注控制
\usepackage{layouts}        % 打印当前页面格式的宏包
\usepackage{paralist}       % 一种换行不缩进的列表格式,用 asparaenum 或者 inparaenum 环境
%\usepackage[numbers]{natbib}         % 参考文献
\usepackage{fancyvrb}       % 原样输出
\usepackage[numbers,sort&compress,square,super]{natbib} %参考文献
\usepackage{hypernat}
\usepackage{bibentry}
\usepackage{type1cm} 
%\usepackage{listings}   % 粘贴源代码
%\usepackage[shortlabels]{enumitem}
% 字体
% 表格处理
\usepackage{booktabs}   % 三线表
\usepackage{multirow}   % 表格多行处理
\usepackage{diagbox}    % 斜线表头
\usepackage{tabularx}   % 表格折行
\usepackage{siunitx}    % 国际单位,小数点对齐

% 图形相关
\usepackage{graphicx}         % 请在引用图片时务必给出后缀名
\usepackage[x11names]{xcolor} % 支持彩色
\usepackage[below]{placeins}  % 浮动图形控制宏包
\usepackage{rotating}	      % 图形和表格的控制
\usepackage{picinpar}
\usepackage{setspace}         % 定制表格和图形的多行标题行距
\usepackage{subfigure}           % 插入子图形
\usepackage[subfigure]{ccaption} % 插图表格的双语标题
%\usepackage{CJK}        % 中文支持宏包
\usepackage{amsmath}    % AMS 宏包,包含一些常用的数学字体和符号,
\usepackage{amssymb}    % 如果你引用一些不常见的字体宏包可能会与此发生冲突,请考虑略去。
\usepackage[mathscr]{eucal} % to replace the following commented package
%\usepackage{mathrsfs}   % 一种常用于定义泛函算子的花体字母,只有大写。
\usepackage[amsmath,thmmarks,hyperref]{ntheorem} % 定理类环境宏包,其中 amsmath 选项用来兼容 AMS的宏包
%\usepackage[amsmath,thmmarks]{ntheorem}
%\usepackage{type1cm}    % 控制字体的大小
\usepackage{bm}         % 处理数学公式中的黑斜体的宏包
%\usepackage{fourier}    % 一种和Times类似的字体
\usepackage{txfonts}   % 如果你希望和 Word 的默认值保持一致,请使用此宏包
\usepackage{helvet}
\newcommand{\BigO}[1]{\ensuremath{\operatorname{O}\bigl(#1\bigr)}}
\usepackage[table]{xcolor}
\newcommand{\me}{\mathrm{e}}

\usepackage[T1]{fontenc}
\usepackage{makecell}
\renewcommand{\rothead}[3][90]{\makebox[9mm][c]{\rotatebox{#1}{\makecell[c]{#2}}}}%
\usepackage{multirow}
\usepackage{rotating,tabularx}

% 图形
\usepackage{graphicx}   % 这里引用 graphicx 宏包时没有图形驱动选项,所以请在引用图片时务必给出后缀名
\usepackage{subfigure}  % 插入子图形
\usepackage{color}      % 支持彩色
\usepackage[below]{placeins}    % 浮动图形控制宏包
\usepackage{picinpar}   % 图文混排用宏包
%\usepackage{floatflt}  % 图文混排用宏包
\usepackage{rotating}   % 图形和表格的控制
\usepackage[subfigure]{ccaption}    % 浮动图形和表格标题样式,支持插图表格的双语标题
\usepackage{setspace}   % 定制表格和图形的多行标题行距!!
\usepackage[ruled,vlined]{algorithm2e}
\usepackage[font={small}]{caption}

% 其他
\usepackage{calc}   % 在 tex 文件中具有一些计算功能,主要用在页面控制。
%\usepackage{ccmap}      % 使pdfLatex生成的文件支持复制等
%\usepackage[pdftex,
            %CJKbookmarks=true,
            %bookmarksnumbered=true,
            %bookmarksopen=true,
            %colorlinks=true,
            %citecolor=black,
            %linkcolor=black,
            %anchorcolor=green,
            %urlcolor=magenta,
            %breaklinks=true
            %]{hyperref} % 生成有书签和超链接的 pdf
\usepackage{listings}         % 源代码展示
\lstset{%
  language=TeX,
  defaultdialect=empty,
  basicstyle=\ttfamily\small,
  backgroundcolor=\color{LightSteelBlue1},
  keywordstyle=\color{blue},
  showspaces=false,
  showstringspaces=false,
  showtabs=false,
  tabsize=2,breakatwhitespace=false,
  columns=flexible}

\usepackage[xetex,
	bookmarksnumbered=true,
	bookmarksopen=true,
	colorlinks=true,
	% pdfborder={0 0 1},
	citecolor=blue,
	linkcolor=blue,
	anchorcolor=green,
	urlcolor=magenta,
	breaklinks=true,
	CJKbookmarks=true,
	]{hyperref}













