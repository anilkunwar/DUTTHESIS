%以下由王辉增加(2012年12月16日),主要是修改英文目录和图标目录
%Revision 1: 目录中的点的稠密程度
\makeatletter
\def\@dotsep{0.3}%就在这儿改,把2改成其他什么,默认是4.5,就是点之间距离为4.5mu
\makeatother



%Revision 2: Formatting Table of Contents in Chinese and English
%Font
\titlecontents{chapter}[10mm]
{\fontsize{12pt}{15pt}}
{\contentslabel{4em}}
{}
{\titlerule*{.}\contentspage}
%
\titlecontents{section}[12mm]
{\fontsize{12pt}{15pt}}
{\contentslabel{2em}}
{}
{\titlerule*{.}\contentspage}
%
\titlecontents{subsection}[23mm]
{\fontsize{12pt}{15pt}}
{\contentslabel{3em}}
{}
{\titlerule*{.}\contentspage}

%Indent
\dottedcontents{chapter}[0.32cm]{\vspace{0.2em}}{1.0em}{5pt}
\dottedcontents{section}[1.32cm]{}{1.8em}{5pt}
\dottedcontents{subsection}[2.32cm]{}{2.7em}{5pt}
\dottedcontents{subsubsection}[3.32cm]{}{3.4em}{5pt}

%Revision 3: Content in English
\makeatletter
\newcommand\engcontentsname{\hfill CONTENTS \hfill}
%\newcommand\engcontentsname{CONTENTS}
\newcommand\tableofengcontents{%
    \if@twocolumn
      \@restonecoltrue\onecolumn
    \else
      \@restonecolfalse
    \fi
    \chapter*{\engcontentsname
        \@mkboth{%
           \MakeUppercase\engcontentsname}{\MakeUppercase\engcontentsname}}%
    \@starttoc{toe}% !!!!Define a new contents!!!!
    \if@restonecol\twocolumn\fi
    }
\newcommand\addengcontents[2]{%
    \addcontentsline{toe}{#1}{\protect\numberline{\csname the#1\endcsname}#2}}
\makeatother
\newcommand\echapter[1]{\addengcontents{chapter}{#1}}
\newcommand\esection[1]{\addengcontents{section}{#1}}
\newcommand\esubsection[1]{\addengcontents{subsection}{#1}}
\newcommand\esubsubsection[1]{\addengcontents{subsubsection}{#1}}


