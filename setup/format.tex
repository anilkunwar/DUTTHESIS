%%%%%%%%%%%%%%%%%%%%%%%%%%%%%%%%%%%%%%%%%%%%%%%%%%%%%%%%%%%%%%%%%%%%%%%%%%%%%%%%
%
%   Doctoral Dissertation of Dalian University of Technology
%   By Jie Li and Ahmedin Mohammed Ahmed on the work of Hui Wang
%   format.tex
%   Version: 0.1
%   Update: 2014-04-16
%   Environment: Windows 8.1 + CTeX 2.9.2.164 + WinEdt 7.0
%   欢迎使用,如有任何意见和建议请发送邮件至 yuri_1985@163.com,感谢您的支持!
%
%%%%%%%%%%%%%%%%%%%%%%%%%%%%%%%%%%%%%%%%%%%%%%%%%%%%%%%%%%%%%%%%%%%%%%%%%%%%%%%%

%%%%%%%%%%%%%%%%%%%%%%%%%%%%%%%%%%%%%%%%%%%%%%%%%%%%%%%%%%%%%%%%%%%%%%%%%%%%%%%%80
% Settings of Pages
% 页面设置
%%%%%%%%%%%%%%%%%%%%%%%%%%%%%%%%%%%%%%%%%%%%%%%%%%%%%%%%%%%%%%%%%%%%%%%%%%%%%%%%80
% A4 纸张
\setlength{\paperwidth}{21.0cm}
\setlength{\paperheight}{29.7cm}

% Size of the content
% 设置正文尺寸大小
\setlength{\textwidth}{16.0cm}
\setlength{\textheight}{23.7cm}

% To make the content in the middle of the page
% 设置正文区在正中间
\newlength \mymargin
\setlength{\mymargin}{(\paperwidth-\textwidth)/2}
\setlength{\oddsidemargin}{(\mymargin)-1in}
\setlength{\evensidemargin}{(\mymargin)-1in}


% 设置正文区偏移量,奇数页向右偏,偶数页向左偏
\newlength \myshift
\setlength{\myshift}{0.35cm}     % 双面打印的奇偶页偏移值,可根据需要修改,建议小于 0.5 cm。
\addtolength{\oddsidemargin}{\myshift}
\addtolength{\evensidemargin}{-\myshift}
\setlength{\topmargin}{-0.75cm}
\setlength{\headheight}{0.50cm}
\setlength{\headsep}{0.90cm}
\setlength{\footskip}{1.47cm}

% Adjustment of equations
% 公式的精调
\allowdisplaybreaks[4]  % 可以让公式在排不下的时候分页排,这可避免页面有大段空白。

%下面这组命令使浮动对象的缺省值稍微宽松一点,从而防止幅度
%对象占据过多的文本页面,也可以防止在很大空白的浮动页上放置
%很小的图形。
\renewcommand{\textfraction}{0.15}
\renewcommand{\topfraction}{0.85}
\renewcommand{\bottomfraction}{0.65}
\renewcommand{\floatpagefraction}{0.60}

%%%%%%%%%%%%%%%%%%%%%%%%%%%%%%%%%%%%%%%%%%%%%%%%%%%%%%%%%%%%%%%%%%%%%%%%%%%%%%%%80
% Font and Size
% 字体字号定义
%%%%%%%%%%%%%%%%%%%%%%%%%%%%%%%%%%%%%%%%%%%%%%%%%%%%%%%%%%%%%%%%%%%%%%%%%%%%%%%%80
% Indent of Chinese
% 中文段落缩进和 ~ 字符替换
%\CJKindent                              % 首行缩进两个汉字
%\sloppy
%\CJKspace                              % 中英文混排的断行
%CJKtilde                               % 重新定义~,用~隔开中英文

% Chinese Font
% 中文字体
\newcommand{\song}{\CJKfamily{song}}    % 宋体 (Font of SONG)
\newcommand{\fs}{\CJKfamily{fs}}        % 仿宋 (Font of FANG SONG)
\newcommand{\kai}{\CJKfamily{kai}}      % 楷体
\newcommand{\hei}{\CJKfamily{hei}}      % 黑体
%\newcommand{\li}{\CJKfamily{li}}        % 隶书
%\newcommand{\you}{\CJKfamily{you}}      % 幼圆
%\newcommand{\xk}{\CJKfamily{STXingkai}}
% 字号

\newcommand{\yihao}{\fontsize{26pt}{39pt}\selectfont}               % 一号,1.5 倍行距
%\newcommand{\twentysix}{\fontsize{26pt}{39pt}\selectfont}               % 一号,1.5 倍行距
\newcommand{\xiaoyi}{\fontsize{24pt}{30pt}\selectfont}              % 小一,1.25倍行距
%\newcommand{\twentyfour}{\fontsize{24pt}{30pt}\selectfont}              % 小一,1.25倍行距
%\newcommand{\erhao}{\fontsize{22pt}{27.5pt}\selectfont}             % 二号,1.25倍行距
\newcommand{\xiaoer}{\fontsize{18pt}{22.5pt}\selectfont}            % 小二,1.25倍行距
\newcommand{\eighteen}{\fontsize{18pt}{27pt}\selectfont}            % 18 pt and 27 pt line space
\newcommand{\sanhao}{\fontsize{16pt}{20pt}\selectfont}              % 三号,1.25倍行距
\newcommand{\xiaosan}{\fontsize{15pt}{18.75pt}\selectfont}          % 小三,1.25倍行距
\newcommand{\daxiaosi}{\fontsize{12pt}{18pt}\selectfont}            % 小四,1.5 倍行距
\newcommand{\ltwelve}{\fontsize{12pt}{18pt}\selectfont}             % large twelve with 12pt, 18pt line space
\newcommand{\sihao}{\fontsize{14pt}{17pt}\selectfont}               % 四号,1.25倍行距
\newcommand{\fourteen}{\fontsize{14pt}{17pt}\selectfont}            % 14 pt and 17 pt line space
\newcommand{\thirteen}{\fontsize{13pt}{19.5pt}\selectfont}          % 13 pt, 19.5 pt line space
\newcommand{\xiaosi}{\fontsize{12pt}{15pt}\selectfont}              % 小四,1.25倍行距
\newcommand{\twelve}{\fontsize{12pt}{15pt}\selectfont}              % 小四(12 pt),1.25倍行距
\newcommand{\eleven}{\fontsize{11pt}{13.75pt}\selectfont}           % 11 pt, 13.75 line space
\newcommand{\dawu}{\fontsize{10.5pt}{18pt}\selectfont}              % 五号,1.75倍行距
\newcommand{\zhongwu}{\fontsize{10.5pt}{15pt}\selectfont}           % 五号,1.5 倍行距
\newcommand{\wuhao}{\fontsize{10.5pt}{10.5pt}\selectfont}           % 五号,单倍行距
\newcommand{\ten}{\fontsize{10pt}{10pt}\selectfont}                 % 五号,单倍行距
\newcommand{\xiaowu}{\fontsize{9pt}{9pt}\selectfont}                % 小五,单倍行距

% defaultfont 默认字体命令
\def\defaultfont{\renewcommand{\baselinestretch}{1.25}
\fontsize{12pt}{15pt}\selectfont}

% Font in Table of Content
% 设置目录字体和行间距
\def\defaultmenufont{\renewcommand{\baselinestretch}{1.22}
\fontsize{12pt}{12pt}}

% Content with fixed distance and underline
% 固定距离内容填入及下划线
\makeatletter
    \newcommand\fixeddistanceleft[2][1cm]{{\hb@xt@ #1{#2\hss}}}
    \newcommand\fixeddistancecenter[2][1cm]{{\hb@xt@ #1{\hss#2\hss}}}
    \newcommand\fixeddistanceright[2][1cm]{{\hb@xt@ #1{\hss#2}}}
    \newcommand\fixedunderlineleft[2][1cm]{\underline{\hb@xt@ #1{#2\hss}}}
    \newcommand\fixedunderlinecenter[2][1cm]{\underline{\hb@xt@ #1{\hss#2\hss}}}
    \newcommand\fixedunderlineright[2][1cm]{\underline{\hb@xt@ #1{\hss#2}}}
\makeatother
%%%%%%%%%%%%%%%%%%%%%%%%%%%%%%%%%%%%%%%%%%%%%%%%%%%%%%%%%%%%%%%%%%%%%%%%%%%%%%%%80
% Subtitles
% 标题相关
%%%%%%%%%%%%%%%%%%%%%%%%%%%%%%%%%%%%%%%%%%%%%%%%%%%%%%%%%%%%%%%%%%%%%%%%%%%%%%%%80
% Definition, Theorem, etc
% 定义、定理等环境
%\theoremstyle{plain}
%\theoremheaderfont{\hei\bf}
%\theorembodyfont{\song\rmfamily}
%\newtheorem{definition}{\hei 定义}[chapter]
%\newtheorem{example}{\hei 例}[chapter]
%\newtheorem{algorithm}{\hei 算法}[chapter]
%\newtheorem{theorem}{\hei 定理}[chapter]
%\newtheorem{axiom}{\hei 公理}[chapter]
%\newtheorem{proposition}[theorem]{\hei 命题}
%\newtheorem{property}{\hei 性质}
%\newtheorem{lemma}[theorem]{\hei 引理}
%\newtheorem{corollary}{\hei 推论}[chapter]
%\newtheorem{remark}{\hei 注解}[chapter]
%\newenvironment{proof}{{\hei证明 }}%
%{\hfill $\square$ \vskip 4mm}
%\theoremsymbol{$\square$}
%%%%%%%%%%%%%%%%%%%%%%%%%%%%%%%%%%%%%%%%%%%%%%%%%%%%%%%%%%
\theoremstyle{plain}
\theoremheaderfont{}
\theorembodyfont{}
\newtheorem{definition}{}[chapter]
\newtheorem{example}{}[chapter]
%\newtheorem{algorithm}{}[chapter]
\newtheorem{theorem}{}[chapter]
\newtheorem{axiom}{}[chapter]
\newtheorem{proposition}[theorem]{}
\newtheorem{property}{}
\newtheorem{lemma}[theorem]{}
\newtheorem{corollary}{}[chapter]
\newtheorem{remark}{}[chapter]
\newenvironment{proof}{}%
{\hfill $\square$ \vskip 4mm}
%\theoremsymbol{$\square$}


% 目录标题
\renewcommand\contentsname{\hfill 中文目录 \hfill}
%\renewcommand\listfigurename{\hfill 图~表~目~录 \hfill} %王辉修改
%\renewcommand\listtablename{\hfill 表~格~目~录 \hfill} %王辉修改
%\renewcommand{\bibname}{\hfill 参~考~文~献 \hfill}
%\renewcommand{\algorithmcfname}{算法}
%\renewcommand{\algorithmcfname}{Algorithm}

%%%%%%%%%%%%%%%%%%%%%%%%%%%%%%%%%%%%%%%%%%%%%%%%%%%%%%%%%%%%%%%%%%%%%%%%%%%%%%%%80
% Chapters, sections
% 段落章节
%%%%%%%%%%%%%%%%%%%%%%%%%%%%%%%%%%%%%%%%%%%%%%%%%%%%%%%%%%%%%%%%%%%%%%%%%%%%%%%%80
\setcounter{secnumdepth}{4}
\setcounter{tocdepth}{4}
% 设置章、节、小节、小小节的间距
\titleformat{\chapter}[hang]{\eighteen}{\eighteen\thechapter}{10pt}{\eighteen}
\titlespacing{\chapter}{0pt}{-3ex  plus .1ex minus .2ex}{3.3ex}
\titleformat{\section}[hang]{\fourteen}{\fourteen\thesection}{0.5em}{}{}
\titlespacing{\section}{0pt}{0.5em}{0.5em}
\titleformat{\subsection}[hang]{\twelve}{\twelve\thesubsection}{0.5em}{}{}
\titlespacing{\subsection}{0pt}{0.5em}{0.3em}
\titleformat{\subsubsection}[hang]{}{\thesubsubsection}{0.5em}{}{}
\titlespacing{\subsubsection}{0pt}{0.3em}{0pt}
% 缩小目录中各级标题之间的缩进
\dottedcontents{chapter}[0.32cm]{\vspace{0.2em}}{1.0em}{5pt}
\dottedcontents{section}[1.32cm]{}{1.8em}{5pt}
\dottedcontents{subsection}[2.32cm]{}{2.7em}{5pt}
\dottedcontents{subsubsection}[3.32cm]{}{3.4em}{5pt}

%
% 段落之间的竖直距离
\setlength{\parskip}{1.2pt}
% Line space
% 定义行距
\renewcommand{\baselinestretch}{1.27}
% Line space of references
% 参考文献条目间行间距
\setlength{\bibsep}{1pt}

%%%%%%%%%%%%%%%%%%%%%%%%%%%%%%%%%%%%%%%%%%%%%%%%%%%%%%%%%%%%%%%%%%%%%%%%%%%%%%%%80
% Head and foot, using package of fancyhdr
% 页眉和页脚 使用 fancyhdr 宏包
%%%%%%%%%%%%%%%%%%%%%%%%%%%%%%%%%%%%%%%%%%%%%%%%%%%%%%%%%%%%%%%%%%%%%%%%%%%%%%%%80

\newcommand{\makeheadrule}{%
    \makebox[0pt][l]{\rule[.7\baselineskip]{\headwidth}{0.5pt}}%
    \vskip-.8\baselineskip}

\makeatletter
\renewcommand{\headrule}{%
    {\if@fancyplain\let\headrulewidth\plainheadrulewidth\fi
     \makeheadrule}}

\pagestyle{fancyplain}

\fancyhf{}
\fancyhead[CO]{\ten Dalian University of Technology Doctoral Dissertation}
\fancyhead[CE]{\ten\@etitle}
\fancyfoot[C,C]{\ten--~\thepage~--}

% Clear Header Style on the Last Empty Odd pages
\makeatletter
\def\cleardoublepage{\clearpage\if@twoside \ifodd\c@page\else%
    \hbox{}%
    \thispagestyle{empty}%              % Empty header styles
    \newpage%
    \if@twocolumn\hbox{}\newpage\fi\fi\fi}



%%%%%%%%%%%%%%%%%%%%%%%%%%%%%%%%%%%%%%%%%%%%%%%%%%%%%%%%%%%%%%%%%%%%%%%%%%%%%%%%80
% 调整列表环境的垂直间距
%%%%%%%%%%%%%%%%%%%%%%%%%%%%%%%%%%%%%%%%%%%%%%%%%%%%%%%%%%%%%%%%%%%%%%%%%%%%%%%%80
\let\orig@Itemize =\itemize
\let\orig@Enumerate =\enumerate
\let\orig@Description =\description

\def\Myspacing{\itemsep=1ex \topsep=-4ex \partopsep=-2ex \parskip=-1ex \parsep=2ex}
\def\newitemsep{
\renewenvironment{itemize}{\orig@Itemize\Myspacing}{\endlist}
\renewenvironment{enumerate}{\orig@Enumerate\Myspacing}{\endlist}
\renewenvironment{description}{\orig@Description\Myspacing}{\endlist}
}
\def\olditemsep{
\renewenvironment{itemize}{\orig@Itemize}{\endlist}
\renewenvironment{enumerate}{\orig@Enumerate}{\endlist}
\renewenvironment{description}{\orig@Description}{\endlist}
}
\renewcommand{\labelenumi}{(\arabic{enumi})}
\newitemsep

% fancyvrb宏包中\VerbatimInput的格式设置
\fvset{fontsize=\small, frame=single, baselinestretch=1}

% 修改引用的格式
\newcommand{\ucite}[1]{$^{\mbox{\scriptsize \citep{#1}}}$} % 增加 \ucite 命令使显示的引用为上标形式
\newcommand{\citeup}[1]{$^{\mbox{\scriptsize \citep{#1}}}$} % for WinEdt users

%参考文献
%\addtolength{\bibsep}{-0.2cm}

%%%%%%%%%%%%%%%%%%%%%%%%%%%%%%%%%%%%%%%%%%%%%%%%%%%%%%%%%%%%%%%%%%%%%%%%%%%%%%%%80
% Figures and tables
% 图形表格
%%%%%%%%%%%%%%%%%%%%%%%%%%%%%%%%%%%%%%%%%%%%%%%%%%%%%%%%%%%%%%%%%%%%%%%%%%%%%%%%80

\renewcommand{\figurename}{Fig.}
\renewcommand{\tablename}{Tab.}
%\captionstyle{\centering}
%\hangcaption
\captiondelim{\hspace{1em}}
%\captionnamefont{\song\rmfamily\zhongwu\selectfont}
%\captiontitlefont{\song\rmfamily\zhongwu\selectfont}

\newcommand{\tablepage}[2]{\begin{minipage}{#1}\vspace{0.5ex} #2 \vspace{0.5ex}\end{minipage}}
\newcommand{\returnpage}[2]{\begin{minipage}{#1}\vspace{0.5ex} #2 \vspace{-1.5ex}\end{minipage}}


%%%%%%%%%%%%%%%%%%%%%%%%%%%%%%%%%%%%%%%%%%%%%%%%%%%%%%%%%%%%%%%%%%%%%%%%%%%%%%%%80
% 定义题头格言的格式
%%%%%%%%%%%%%%%%%%%%%%%%%%%%%%%%%%%%%%%%%%%%%%%%%%%%%%%%%%%%%%%%%%%%%%%%%%%%%%%%80

%\newsavebox{\AphorismAuthor}
%\newenvironment{Aphorism}[1]
%{\vspace{0.5cm}\begin{sloppypar} \slshape
%\sbox{\AphorismAuthor}{#1}
%\begin{quote}\small\itshape }
%{\\ \hspace*{\fill}------\hspace{0.2cm} \usebox{\AphorismAuthor}
%\end{quote}
%\end{sloppypar}\vspace{0.5cm}}

%自定义一个空命令,用于注释掉文本中不需要的部分。
\newcommand{\comment}[1]{}

% This is the flag for longer version
\newcommand{\longer}[2]{#1}

\newcommand{\ds}{\displaystyle}

% define graph scale
\def\gs{1.0}

%%%%%%%%%%%%%%%%%%%%%%%%%%%%%%%%%%%%%%%%%%%%%%%%%%%%%%%%%%%%%%%%%%%%%%%%%%%%%%%%80
% 自定义项目列表标签及格式 \begin{denselist} 列表项 \end{denselist}
%%%%%%%%%%%%%%%%%%%%%%%%%%%%%%%%%%%%%%%%%%%%%%%%%%%%%%%%%%%%%%%%%%%%%%%%%%%%%%%%80
\newcounter{newlist} %自定义新计数器
\newenvironment{denselist}[1][temp]{%%%%%定义新环境:可改变的列表题目
\begin{list}{\textbf{#1} (\arabic{newlist})} %%标签格式
    {
    \usecounter{newlist}
     \setlength{\labelwidth}{22pt} %标签盒子宽度
     \setlength{\labelsep}{0cm} %标签与列表文本距离
     \setlength{\leftmargin}{0cm} %左右边界
     \setlength{\rightmargin}{0cm}
     \setlength{\parsep}{0ex} %段落间距
     \setlength{\itemsep}{0ex} %标签间距
     \setlength{\itemindent}{44pt} %标签缩进量
     \setlength{\listparindent}{44pt} %段落缩进量
    }}
{\end{list}}%%%%%

%%%%%%%%%%%%%%%%%%%%%%%%%%%%%%%%%%%%%%%%%%%%%%%%%%%%%%%%%%%%%%%%%%%%%%%%%%%%%%%%80
% cover
% 封面摘要
%%%%%%%%%%%%%%%%%%%%%%%%%%%%%%%%%%%%%%%%%%%%%%%%%%%%%%%%%%%%%%%%%%%%%%%%%%%%%%%%80
%\def\cdegree#1{\def\@cdegree{#1}}\def\@cdegree{}
\def\ctitle#1{\def\@ctitle{#1}}\def\@ctitle{}
%\def\etitle#1{\def\@etitle{#1}}\def\@etitle{}
\def\department#1{\def\@department{#1}}\def\@department{}
\def\caffil#1{\def\@caffil{#1}}\def\@caffil{}
\def\csubject#1{\def\@csubject{#1}}\def\@csubject{}
\def\cauthor#1{\def\@cauthor{#1}}\def\@cauthor{}
\def\cauthorno#1{\def\@cauthorno{#1}}\def\@cauthorno{}
\def\csupervisor#1{\def\@csupervisor{#1}}\def\@csupervisor{}
\def\cdate#1{\def\@cdate{#1}}\def\@cdate{}
\long\def\cabstract#1{\long\def\@cabstract{#1}}\long\def\@cabstract{}
\def\ckeywords#1{\def\@ckeywords{#1}}\def\@ckeywords{}
\def\etitle#1{\def\@etitle{#1}}\def\@etitle{}
\long\def\eabstract#1{\long\def\@eabstract{#1}}\long\def\@eabstract{}
\def\ekeywords#1{\def\@ekeywords{#1}}\def\@ekeywords{}



\def\eoriginality{
    \renewcommand{\baselinestretch}{1.61}
    % Originality Declaration
    \newpage
    \thispagestyle{empty}
    \begin{center}
    \eighteen {Dalian University of Technology Dissertation \\ Originality Declaration}
    \end{center}
    \parbox[t][7.68cm][c]{\textwidth}{\twelve
            \noindent
             I declare that this dissertation is the result of an independent research I have made under the supervision of my supervisor. It does not contain any published or unpublished works or research results by other individuals or institutions apart from those that have been referenced in the form of references or notes. All individuals and institutions that have made contributions to my research have been acknowledged in the Acknowledgement.

           \noindent
           % \vspace{0.47cm}
            I am fully aware that I myself will bear all the legal responsibility arising from violation of the above declaration. \\
            ~~~~~~\\
        }
        \noindent

        {\twelve Dissertation Title: \fixedunderlinecenter[12.6cm]{\@etitle}}\\
        ~~\\

        {\twelve Author Signature: \fixeddistanceleft[12.6cm]{\underline{\hspace{7.3cm}}\hfill Date: \underline{\hspace{1.1cm}}/{\underline{\hspace{1.6cm}}/{\underline{\hspace{1.1cm}}}}}}\\

    \renewcommand{\baselinestretch}{1.27}
    \cleardoublepage
}

\def\originality{
    \renewcommand{\baselinestretch}{1.61}
    % 独创性说明
    \newpage
    \thispagestyle{empty}
    \begin{center}
        \parbox[t][1.52cm][c]{\textwidth}{{\xiaoer\song\centerline{大连理工大学学位论文独创性声明}}}
        \parbox[t][7.68cm][c]{\textwidth}{\sihao\fs
            \noindent
            作者郑重声明:所呈交的学位论文,是本人在导师的指导下进行研究工作所取得的成果。%
            尽我所知,除文中已经注明引用内容和致谢的地方外,%
            本论文不包含其他个人或集体已经发表的研究成果,%
            也不包含其他已申请学位或其他用途使用过的成果。%
            与我一同工作的同志对本研究所做的贡献均已在论文中做了明确的说明并表示了谢意。%

            \noindent
            \vspace{0.47cm}
            若有不实之处,本人愿意承担相关法律责任。
        }
        \noindent
        %\parbox[t][0.58cm][c]{\textwidth}
        {\sihao\fs 学{\hfill}位{\hfill}论{\hfill}文{\hfill}题{\hfill}目{\hfill}:\fixedunderlinecenter[12.6cm]{\@ctitle}}\\
        \sihao\fs~~~\\
        %\parbox[t][1.22cm][c]{\textwidth}
        {\sihao\fs 作{\hfill}者{\hfill}签{\hfill}名{\hfill}:\fixeddistanceleft[12.6cm]{\underline{\hspace{5.8cm}}\hfill
	       日期:\underline{\hspace{1.4cm}}~年~{\underline{\hspace{0.7cm}}~月~{\underline{\hspace{0.7cm}}~日~}}}}\\
    \end{center}
    \renewcommand{\baselinestretch}{1.27}

    \cleardoublepage
}

\def\makecover{
    \begin{titlepage}
    % Cover
        \newpage
        \thispagestyle{empty}
            \begin{center}
            \parbox[t][3cm][c]{12.8cm}
            {
                \begin{center}
                    {\eighteen {\@etitle}\\}
                    {\eighteen{\textit{\@ctitle}}\\}
                    %{\eighteen{( \textit{\@ctitle})}\\}
                \end{center}
            }
            \parbox[c][16.8cm][c]{\textwidth}
            {

                \begin{center}

                    {~~\\
                    \twelve{by} \\
                    ~~~\\
                    }
                    {\defaultfont {\@cauthor}\\}
                    {\defaultfont(\textit{\@cauthorno})\\}
                    {~~\\
                    \twelve {to} \\
                    ~~~\\
                    }
                    {\twelve {\textit{\@department}}\\}
                    {~~~\\
                    \defaultfont {in partial fulfillment of the requirements}\\
                    \defaultfont {for the degree of}\\
                    \defaultfont {Doctor of Philosophy}\\
                    \defaultfont {in the subject of}\\
                    }
                    {\defaultfont {\textit{\@csubject}}\\}
                    {~~~\\
                    \defaultfont {on}\\}
                    {\defaultfont {\textit{\@cdate}}\\}
                    {~~\\
                    \defaultfont {Dissertation Supervisor}\\
                    ~~~\\}
                    {\defaultfont{\textit{\@csupervisor}}\\}
                    {~~\\
                    ~~\\
                    ~~~\\}
                    %\end{tabular}
                \end{center}

            }

            {
                \begin{figure}[h]
                    \centering
                    \includegraphics[scale=1]{DUT.pdf}
                \end{figure}
                \vspace{-0.35cm}
                \defaultfont{Dalian University of Technology}
            }
            \end{center}
        \cleardoublepage
    \end{titlepage}
}

\def\eauthorization
{
\newpage
\begin{center}
    \eighteen Dalian University of Technology Doctoral Dissertation Copyright Use Authorization
\end{center}

\twelve
I fully understand relevant regulations regarding university dissertation copyright. Copyright of theses at university during PhD period belongs to Dalian University of Technology, allowing theses to be consulted and borrowed. University has right to retain theses and submit copies and electronic editions to national departments and institutions concerned. University can index relevant database for retrieval using part or the whole of this dissertation. University can photocopy, print in reduced format or scan to keep and compile this dissertation.
\\

\noindent
Dissertation Title: \fixedunderlinecenter[12.6cm]{\@etitle}\\
~~\\
Author's Signature: \fixeddistanceleft[12.6cm]{\underline{\hspace{7.3cm}}\hfill
    Date: \underline{\hspace{1.1cm}}/{\underline{\hspace{1.6cm}}/{\underline{\hspace{1.1cm}}}}}\\
    ~~~\\
Supervisor's Signature: \fixeddistanceleft[12.0cm]{\underline{\hspace{6.7cm}}\hfill
	 Date: \underline{\hspace{1.1cm}}/{\underline{\hspace{1.6cm}}/{\underline{\hspace{1.1cm}}}}}\\
\renewcommand{\baselinestretch}{1.27}
}

\def\authorization
{
\newpage
\begin{center}
    \eighteen\song~大连理工大学学位论文版权使用授权书~
\end{center}

\sihao\fs\noindent
  本人完全了解学校有关学位论文知识产权的规定,%
在校攻读学位期间论文工作的知识产权属于大连理工大学,允许论文被查阅和借阅。%
学校有权保留论文并向国家有关部门或机构送交论文的复印件和电子版,%
可以将本学位论文的全部或部分内容编入有关数据库进行检索,%
可以采用影印、缩印、或扫描等复制手段保存和汇编本学位论文。%
\\

\noindent
学{\hfill}位{\hfill}论{\hfill}文{\hfill}题{\hfill}目{\hfill}:\fixedunderlinecenter[12.6cm]{\@ctitle}\\
作{\hfill}者{\hfill}签{\hfill}名{\hfill}:\fixeddistanceleft[12.6cm]{\underline{\hspace{6.0cm}}\hfill
    ~~日期:\underline{\hspace{1.4cm}}~年~{\underline{\hspace{0.7cm}}~月~{\underline{\hspace{0.7cm}}~日~}}}\\
导{\hfill}师{\hfill}签{\hfill}名{\hfill}:\fixeddistanceleft[12.6cm]{\underline{\hspace{6.0cm}}\hfill
	 ~~日期:\underline{\hspace{1.4cm}}~年~{\underline{\hspace{0.7cm}}~月~{\underline{\hspace{0.7cm}}~日~}}}\\
\renewcommand{\baselinestretch}{1.27}
}

\def\makeabstract{
    \defaultfont

    \chapter*{\hfill Abstract \hfill}
    %\addcontentsline{toe}{chapter}{Abstract}
    \setcounter{page}{1}
    \@eabstract
    \vspace{0.53cm}

    {\textbf{Keywords:} \@ekeywords}

    \defaultfont
    \cleardoublepage
    \chapter*{}
    \vspace{-1.38cm}
    \vspace{-0.36cm}
    \begin{center}
    {
       \eighteen\song{摘~~~~~~~~要}\\
    }
    \end{center}
    \vspace{0.11cm}

    \@cabstract

    \vspace{0.53cm}

    {\hei{关键词:{\fs\@ckeywords}}}

    \defaultfont
    \cleardoublepage
}

\makeatletter
\def\hlinewd#1{%
  \noalign{\ifnum0=`}\fi\hrule \@height #1 \futurelet
   \reserved@a\@xhline}
\makeatother

%定义索引生成
\def\generateindex
{
    \addcontentsline{toc}{chapter}{\indexname}
    \printindex
    \cleardoublepage
}

\raggedbottom

% 格式文件结束
