%%%%%%%%%%%%%%%%%%%%%%%%%%%%%%%%%%%%%%%%%%%%%%%%%%%%%%%%%%%%%%%%%%%%%%%%%%%%%%%%
% % !TEX TS-program = XeLaTeX
% !TEX encoding = UTF-8 Unicode

%%%%%%%%%%%%%%%%%%%%%%%%%%%%%%%%%%%%%%%%%%%%%%%%%%%%%%%%%%%%%%%%%%%%%% 
% 
%   Doctoral Dissertation of Dalian University of Technology
%   By Anil Kunwar on the works of Jie Li, Ahmedin Mohammed Ahmed, Hui Wang, Yuri and Whufanwei
%   main.tex
%   Version: 0.1
%   Update: 2014-04-16
%% Environment 编译环境1:Ubuntu 14.04 + TeXLive 2015 
%   For further details,contact yuri_1985@163.com
%
%%%%%%%%%%%%%%%%%%%%%%%%%%%%%%%%%%%%%%%%%%%%%%%%%%%%%%%%%%%%%%%%%%%%%%%%%%%%%%%%

\etitle{Dimensional Change of Solder Bubbles and Intermetallic Compounds under Field Gradients in Tin Solders}
\ctitle{大连理工大学博士学位论文~\XeLaTeX{}~模版}
%\ctitle{大连理工大学博士学位论文格式规范}

\cauthor{Anil Kunwar}
\cauthorno{11225045}
\department{School of Materials Science and Engineering}
\csubject{Materials Science}


%\cdate{November 22, 2014}
\cdate{February 5, 2016}
\csupervisor{Ma Haitao}

\eabstract{
Abstract and in English Language
\paragraph*{}As the electronic packaging industries are initiating their shift towards miniaturization and lead free soldering, the reliability of the solder joints has been a topic of significant concern among the researchers. For a liquid tin based solder attached to a copper substrate or sandwiched beteen the substrates, the interface of the solder-substrate system becomes the site for planar microbubbles and intermetallic compounds (IMCs). The solder bubbles or voids reduce the effective area of the solder joint surface or section and thus decrease the net strength of the material. Moreover, these voids act as the site of stress concentration and are a risk factor for the failure of the joints and connections. IMCs, more prevalently, Cu$_6$Sn$_5$ are brittle in nature and their thickness is considered a factor affecting the overall reliability of the solder-substrate system. Both bubbles and IMCs, occurring at the interface of the material, characterize the vulnerability of interfacial zone of the solder-substrate system. The study of voids and IMCs in lead free solder - substrate system provides a unified approach for the overall interfacial reliability  in the material.  The use of accurate experimental and numerical model in the assessment of interfacial voids and IMCs can enable in the deduction of useful data and interference.
\paragraph*{}The understanding of the growth mechanism of the bubbles might aid the researchers in finding the solutions such as minimize their occurrence and prevalence in Sn-based solder materials. In the present research, the diffusion-driven growth of a  pre-existing bubble in liquid solder is studied at a temperature of 250 $^o$C. 
}

\ekeywords{5-6 Keywords in English Language}


\cabstract{
Abstract in Chinese Language
}

\ckeywords{Keywords in Chinese Language}


%\XeLaTeX{} Template}
\makecover
